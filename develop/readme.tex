\documentclass[11pt, a4paper]{jsarticle}
%-----%-----%-----%-----%-----%-----%-----%-----%-----%-----%-----%-----%-----%-----%-----%
%\usepackage{--}
\usepackage{amsmath,ascmac,amssymb}
\usepackage{bm}				%\bm{--}
\usepackage{braket}			%\braket{--|--}
\usepackage{comment}			%\begin{comment}--\end{comment}
\usepackage{mathrsfs}			%\mathscr{--}
\usepackage{ulem}				%underline

\usepackage{layout}			%\layout
\usepackage[dvipdfmx]{graphicx}
\usepackage{epsfig}
\usepackage{color}
\newcommand{\Comment}[1]{\textcolor{red}{[[#1]]}}		%\Comment{--} -> [[--]]
\newcommand{\im}[1]{\text{Im}#1}
\newcommand{\re}[1]{\text{Re}#1}
%-----%-----%-----%-----%-----%-----%-----%-----%-----%-----%-----%-----%-----%-----%-----%
\title{
有限群解析プログラムの説明書
}
\author{
作成者: 立石卓也
}

\begin{document}
\newcommand{\Slash}[1]{{\ooalign{\hfil/\hfil\crcr$#1$}}}		%\Slash{--}
%-----%-----%-----%-----%-----%-----%-----%-----%-----%-----%-----%-----%-----%-----%-----%
\maketitle
%-----%-----%-----%-----%-----%-----%-----%-----%-----%-----%-----%-----%-----%-----%-----%
\section{コマンド一覧}
解析画面で用いるコマンドを記す.
全てのコマンドは「コマンド名[引数]」となっている.


\paragraph{Element[group]}
``group''の要素を表示する.

\paragraph{Table[group]}
``group''の乗積表を表示する.
ただし, 位数が大きい場合には一部のみを表示する.

\paragraph{ConjugacyClass[group]}
``group''の共役類を「位数 要素数 [要素の一覧]」として表示する.




%-----%-----%-----%-----%-----%-----%-----%-----%-----%-----%-----%-----%-----%-----%-----%
\end{document}